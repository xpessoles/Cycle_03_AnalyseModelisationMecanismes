\documentclass[10pt,fleqn]{article} % Default font size and left-justified equations
\usepackage[%
    pdftitle={CIN : Cinématique du solide},
    pdfauthor={Xavier Pessoles}]{hyperref}
    
\input{style/new_style}
\input{style/macros_SII}

\usepackage{multicol}
\fichetrue
%\fichefalse

\proftrue
%\proffalse

\tdtrue
%\tdfalse

\courstrue
\coursfalse

\def\discipline{Sciences \\Industrielles de \\ l'Ingénieur}
\def\xxtete{Sciences Industrielles de l'Ingénieur}

\def\classe{PTSI}
\def\xxnumpartie{Cycle 3}
\def\xxpartie{Étude cinématique des systèmes de solides de la chaîne d'énergie  \\
Analyser, Modéliser, Résoudre}

\def\xxnumchapitre{Chapitre 2}
\def\xxchapitre{}%Étude des chaînes fermées : Détermination des lois Entrée -- Sortie}

\def\xxtitreexo{Robinet à pointeau}
\def\xxsourceexo{\hspace{.2cm} D'après ressources de JP Pupier.}


\def\xxposongletx{2}
\def\xxposonglettext{1.45}
\def\xxposonglety{20}
\def\xxonglet{Part. 1 -- Ch. 3}

\def\xxactivite{Colle 1}
\def\xxauteur{\textsl{Xavier Pessoles}}

\def\xxcompetences{%
\textsl{%
\textbf{Savoirs et compétences :}\\
%\noindent \textbf{Résoudre :} à partir des modèles retenus :
\begin{itemize}[label=\ding{112},font=\color{ocre}] 
\item \textit{Mod2 -- C12} : modélisation cinématique des liaisons entre solides;
\item \textit{Mod2 -- C14} : modèle cinématique d’un mécanisme;
\item \textit{Com1 -- C2} : schémas cinématique, d’architecture, technologique.
\end{itemize}
}}

\def\xxfigures{
%\includegraphics[width=.8\textwidth]{images/prot_01}
}%figues de la page de garde

\def\xxpied{%
Partie 3 -- Étude cinématique des systèmes  \\
Ch 5 : Cinématique du solide -- \xxactivite%
}


\setcounter{secnumdepth}{5}
%---------------------------------------------------------------------------


\begin{document}
%\chapterimage{png/Fond_Cin}
\input{style/new_pagegarde}
\vspace{10cm}
\pagestyle{fancy}
\thispagestyle{plain}


\def\columnseprulecolor{\color{ocre}}
\setlength{\columnseprule}{0.4pt} 
\begin{multicols}{2}
\subsection*{Présentation}

Les diagrammes suivantes proposent une description sommaire d'un robinet à pointeau.
\begin{center}
\includegraphics[width=\linewidth]{images/blocs}
\includegraphics[width=\linewidth]{images/Exigences}
\end{center}

\subsection*{Étude technologique}
\subparagraph{}\textit{Donner la fonction de la pièce 6 et son nom.}
\subparagraph{}\textit{Donner le mouvement relatif de la pièce 5 par rapport à la pièce 4.}
\subparagraph{}\textit{Donner le rôle des pièces 9, 8, 7. }
\subparagraph{}\textit{Quel est l’avantage d’avoir un siège (la pièce 2) démontable ?}
%\subparagraph{}\textit{Il y a sur le dessin une remarque « trou à déplacer à gauche ». Quelle est l’utilité de ce trou et pourquoi cette remarque ?
\subparagraph{}\textit{La pièce 3 est hachurée de façon particulière : quelle est la signification ?}
\subparagraph{}\textit{En quoi le montage de la pièce 10 sur la pièce 3 peut-il être amélioré ?}
\subparagraph{}\textit{Pourquoi la pièce 3 n’est pas de diamètre constant sur toute sa longueur ?}

\subsection*{Étude cinématique}
\subparagraph{}\textit{En utilisant les liaisons élémentaires et normalisées vues en cours donner le schéma cinématique représentant le fonctionnement du robinet à pointeau.}

\end{multicols}


\begin{center}
\includegraphics[width=\linewidth]{images/plan_robinet}
\end{center}

\end{document}


