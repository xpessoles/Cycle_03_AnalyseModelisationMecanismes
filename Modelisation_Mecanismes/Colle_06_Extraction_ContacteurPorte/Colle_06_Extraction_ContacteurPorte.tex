\documentclass[10pt,fleqn]{article} % Default font size and left-justified equations
\usepackage[%
    pdftitle={CIN : Cinématique du solide},
    pdfauthor={Xavier Pessoles}]{hyperref}
    
\input{style/new_style}
\input{style/macros_SII}

\usepackage{multicol}
\fichetrue
%\fichefalse

%\proftrue
\proffalse

\tdtrue
%\tdfalse

\courstrue
\coursfalse

\def\discipline{Sciences \\Industrielles de \\ l'Ingénieur}
\def\xxtete{Sciences Industrielles de l'Ingénieur}

\def\classe{PTSI}
\def\xxnumpartie{Cycle 3}
\def\xxpartie{Analyse et modélisation des contacts entre solides}

\def\xxnumchapitre{Chapitre 1}
\def\xxchapitre{Présentation des outils de la communication technique}%Étude des chaînes fermées : Détermination des lois Entrée -- Sortie}

\def\xxtitreexo{Contacteur de porte}
\def\xxsourceexo{\hspace{.2cm} D'après ressources de JP Pupier.}


\def\xxposongletx{2}
\def\xxposonglettext{1.45}
\def\xxposonglety{20}
\def\xxonglet{Part. 3 -- Ch. 1}

\def\xxactivite{Colle 6}
\def\xxauteur{\textsl{Xavier Pessoles}}

\def\xxcompetences{%
\textsl{%
%\textbf{Savoirs et compétences :}\\
%%\noindent \textbf{Résoudre :} à partir des modèles retenus :
%\begin{itemize}[label=\ding{112},font=\color{ocre}] 
%\item \textit{Mod2 -- C12} : modélisation cinématique des liaisons entre solides;
%\end{itemize}
}}

\def\xxfigures{
\includegraphics[width=.8\textwidth]{images/fig_01}
}%figues de la page de garde

\def\xxpied{%
Cycle 3 -- Analyse et modélisation des contacts entre solides \\
Ch. 1 : Présentation des outils de la communication technique -- \xxactivite%
}


\setcounter{secnumdepth}{5}
%---------------------------------------------------------------------------


\begin{document}
%\chapterimage{png/Fond_Cin}
\input{style/new_pagegarde}
\vspace{7cm}
\pagestyle{fancy}
\thispagestyle{plain}


\def\columnseprulecolor{\color{ocre}}
\setlength{\columnseprule}{0.4pt} 

\ifprof
\begin{center}
\includegraphics[width=.8\linewidth]{images/corr_01}
\includegraphics[width=.8\linewidth]{images/corr_02}
\end{center}

\else
\begin{multicols}{2}
\subsection*{Mise en situation}
Cet appareil s’intègre dans un dispositif de sécurité destiné à donner l’information «porte fermée» sur un train urbain.

Le dessin d’ensemble fourni représente le poussoir. Il permet de faire la liaison entre la ferrure en pente et le contacteur avec une possibilité de réglage aisé du déclenchement électrique.

%Sur un  format A4  d’axe vertical, on définit deux colonnes d’égale largeur, une pour la pièce 1, l’autre pour la pièce 2.

La vue qui sera considérée comme vue de face dans les deux cas est la vue correspondant à la coupe AA du dessin d’ensemble (celle située à gauche du dessin d’ensemble) .

\subparagraph{}\textit{Dessiner la pièce 1 : vue de face en demi-coupe AA (partie coupée à gauche par rapport à l’axe vertical de la pièce), et vue de dessous.}

\subparagraph{}\textit{Dessiner la pièce 2 : vue de face et vue de dessous.}

\begin{center}
\begin{tabular}{|c|c|p{.6\linewidth}|}
\hline
9	&1&	Goupille mécanindus diamètre 4 long 25\\ \hline
8	&1&	\\ \hline
7	&1&	Écrou H M6\\ \hline
6	&1&	 \\ \hline
5	&1&	Ressort 7 spires\\ \hline
4	&1&	Axe \\ \hline
3	&1&	Roulette \\ \hline
2	&1&	Tige \\ \hline
1	&1&	Corps \\ \hline
\hline
\textbf{Rep} & \textbf{Nbr} & \textbf{Désignation} \\ \hline
\end{tabular}
\end{center}


\end{multicols}


\begin{center}
\includegraphics[width=.8\linewidth]{images/fig_02}
\end{center}
\fi
\end{document}


