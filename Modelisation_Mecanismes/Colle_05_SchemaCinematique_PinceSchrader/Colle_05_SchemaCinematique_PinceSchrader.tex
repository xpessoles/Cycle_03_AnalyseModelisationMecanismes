\documentclass[10pt,fleqn]{article} % Default font size and left-justified equations
\usepackage[%
    pdftitle={CIN : Cinématique du solide},
    pdfauthor={Xavier Pessoles}]{hyperref}
    
\input{style/new_style}
\input{style/macros_SII}

\usepackage{multicol}
\fichetrue
%\fichefalse

\proftrue
%\proffalse

\tdtrue
%\tdfalse

\courstrue
\coursfalse

\def\discipline{Sciences \\Industrielles de \\ l'Ingénieur}
\def\xxtete{Sciences Industrielles de l'Ingénieur}

\def\classe{PTSI}
\def\xxnumpartie{Cycle 3}
\def\xxpartie{Étude cinématique des systèmes de solides de la chaîne d'énergie  \\
Analyser, Modéliser, Résoudre}

\def\xxnumchapitre{Chapitre 2}
\def\xxchapitre{}%Étude des chaînes fermées : Détermination des lois Entrée -- Sortie}

\def\xxtitreexo{Pince Schrader}
\def\xxsourceexo{}%\hspace{.2cm} D'après ressources de JP Pupier.}


\def\xxposongletx{2}
\def\xxposonglettext{1.45}
\def\xxposonglety{20}
\def\xxonglet{Part. 1 -- Ch. 3}

\def\xxactivite{Colle 5}
\def\xxauteur{\textsl{Xavier Pessoles}}

\def\xxcompetences{%
\textsl{%
\textbf{Savoirs et compétences :}\\
%\noindent \textbf{Résoudre :} à partir des modèles retenus :
\begin{itemize}[label=\ding{112},font=\color{ocre}] 
\item \textit{Mod2 -- C12} : modélisation cinématique des liaisons entre solides;
\item \textit{Mod2 -- C14} : modèle cinématique d’un mécanisme;
\item \textit{Com1 -- C2} : schémas cinématique, d’architecture, technologique.
\end{itemize}
}}

\def\xxfigures{
%\includegraphics[width=.8\textwidth]{images/prot_01}
}%figues de la page de garde

\def\xxpied{%
Partie 3 -- Étude cinématique des systèmes  \\
Ch 5 : Cinématique du solide -- \xxactivite%
}


\setcounter{secnumdepth}{5}
%---------------------------------------------------------------------------


\begin{document}
%\chapterimage{png/Fond_Cin}
\input{style/new_pagegarde}
\vspace{10cm}
\pagestyle{fancy}
\thispagestyle{plain}


\def\columnseprulecolor{\color{ocre}}
\setlength{\columnseprule}{0.4pt} 
\begin{multicols}{2}
\subsection*{Mise en situation}
La pince ci-contre est la pince de préhension d’un bras manipulateur utilisé pour déplacer des objets d’un poste à l’autre. Il s’agit d’une pince pneumatique simple effet (Fermeture par une commande pneumatique ouverture automatique par ressort). 

Cette pince est munie d’un capteur informant la partie commande du robot de la position de la pince.

Cette pince est décrite par un dessin d’ensemble en fin de document.


\subsection*{Travail à réaliser}
\subparagraph{}\textit{Comprendre le fonctionnement de la pince.}

\subparagraph{}\textit{Sur le premier dessin d’ensemble et à l’aide de la nomenclature identifier et colorier chacune des pièces.}

\subparagraph{}\textit{Définir les sous-ensembles cinématiques : pour cela colorier avec des couleurs différentes le deuxième dessin d’ensemble.}

\subparagraph{}\textit{Définir les liaisons entre ces sous-ensembles.}

\subparagraph{}\textit{Tracer le schéma cinématique minimal en représentation plane.}



\subparagraph{}\textit{Décrire le fonctionnement du coupe tube.}

\subparagraph{}\textit{Tracer le schéma cinématique. Pour cela :
\begin{itemize}
\item identifier les classes d’équivalence cinématique;
\item tracer le graphe de structure;
\item tracer le schéma cinématique.
\end{itemize}}

\begin{center}
\begin{tabular}{|c|c|c|}
\hline
21	&4&	Anneau élastique	 \\ \hline
20	&1&	Doigt inférieur	9	 \\ \hline
19	&1&	Doigt supérieur	 \\ \hline
18	&1&	Axe de doigts 19	 \\ \hline
17	&2&	Ressort	6 \\ \hline
16	&1&	Axe de doigts 20 \\ \hline
15	&1&	Biellette inférieure	 \\ \hline
14	&1&	Axe de biellette 15	 \\ \hline
13	&1&	Biellette supérieure	 \\ \hline
12	&1&	Axe de biellette 13	 \\ \hline
11	&1&	Axe du piston \\ \hline
10	&1&	Rondelle frein \\ \hline
9	&1&	Écrou Hm M 8 \\ \hline
8	&1&	Capteur fin de course \\ \hline
7	&1&	Piston \\ \hline
6	&1&	Joint d’étanchéité \\ \hline
5	&1&	Raccord d’arrivée d’air \\ \hline
4	&1&	Joint torique \\ \hline
3	&1&	Couvercle \\ \hline
2	&1&	Anneau élastique \\ \hline
1	&1&	Corps \\ \hline
\hline
\textbf{Rep} & \textbf{Nbr} & \textbf{Désignation} \\ \hline
\end{tabular}
\end{center}


\end{multicols}


\begin{center}
\includegraphics[width=\linewidth]{images/plan}
\end{center}

\end{document}


