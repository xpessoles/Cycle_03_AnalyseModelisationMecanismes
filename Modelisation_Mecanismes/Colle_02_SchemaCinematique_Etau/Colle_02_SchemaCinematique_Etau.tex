\documentclass[10pt,fleqn]{article} % Default font size and left-justified equations
\usepackage[%
    pdftitle={CIN : Cinématique du solide},
    pdfauthor={Xavier Pessoles}]{hyperref}
    
\input{style/new_style}
\input{style/macros_SII}

\usepackage{multicol}
\fichetrue
%\fichefalse

\proftrue
%\proffalse

\tdtrue
%\tdfalse

\courstrue
\coursfalse

\def\discipline{Sciences \\Industrielles de \\ l'Ingénieur}
\def\xxtete{Sciences Industrielles de l'Ingénieur}

\def\classe{PTSI}
\def\xxnumpartie{Cycle 3}
\def\xxpartie{Étude cinématique des systèmes de solides de la chaîne d'énergie  \\
Analyser, Modéliser, Résoudre}

\def\xxnumchapitre{Chapitre 2}
\def\xxchapitre{}%Étude des chaînes fermées : Détermination des lois Entrée -- Sortie}

\def\xxtitreexo{Étau orientable}
\def\xxsourceexo{\hspace{.2cm} D'après ressources de JP Pupier.}


\def\xxposongletx{2}
\def\xxposonglettext{1.45}
\def\xxposonglety{20}
\def\xxonglet{Part. 1 -- Ch. 3}

\def\xxactivite{Colle 2}
\def\xxauteur{\textsl{Xavier Pessoles}}

\def\xxcompetences{%
\textsl{%
\textbf{Savoirs et compétences :}\\
%\noindent \textbf{Résoudre :} à partir des modèles retenus :
\begin{itemize}[label=\ding{112},font=\color{ocre}] 
\item \textit{Mod2 -- C12} : modélisation cinématique des liaisons entre solides;
\item \textit{Mod2 -- C14} : modèle cinématique d’un mécanisme;
\item \textit{Com1 -- C2} : schémas cinématique, d’architecture, technologique.
\end{itemize}
}}

\def\xxfigures{
%\includegraphics[width=.8\textwidth]{images/prot_01}
}%figues de la page de garde

\def\xxpied{%
Partie 3 -- Étude cinématique des systèmes  \\
Ch 5 : Cinématique du solide -- \xxactivite%
}


\setcounter{secnumdepth}{5}
%---------------------------------------------------------------------------


\begin{document}
%\chapterimage{png/Fond_Cin}
\input{style/new_pagegarde}
\vspace{10cm}
\pagestyle{fancy}
\thispagestyle{plain}


\def\columnseprulecolor{\color{ocre}}
\setlength{\columnseprule}{0.4pt} 
\begin{multicols}{2}
\subsection*{Mise en situation}

Le dessin d’ensemble est donné en fin de paragraphe.
Ce mécanisme, monté sur la table d’une perceuse, permet :
\begin{itemize}
\item le serrage de la pièce à usiner entre les mors 5 et 6 ;
\item l’orientation de la pièce par rapport à l’axe de la perceuse.
\end{itemize}
Cette orientation se fait par rotation autour de deux axes $\left(O,\vect{x}\right)$ et $\left(O,\vect{z}\right)$ conformément au schéma ci-dessous.

\begin{center}
\includegraphics[width=\linewidth]{images/fig_01}
\end{center}


\subsection*{Description complémentaire du mécanisme}
\begin{itemize}
\item Les mors rapportés 5 et 6 sont fixés rigidement par vis respectivement sur les mors 4 et 9.
\item L’écrou en bronze 10 est monté par assemblage forcé sur 9 (liaison complète).
\item La douille en bronze 13 et le cylindre 18sont montés par assemblage forcé sur 4 (ajustement serré).
\item La liaison de 4/2 peut être rendue temporairement complète par serrage (solution non représentée sur le
dessin).
\item La désignation « H7g6 » indique que le mouvement de translation est possible entre deux pièces. 
\end{itemize}

\subsection*{Travail à réaliser}

\subparagraph{}\textit{Le volant 14 est en liaison complète par rapport à la vis 11. Donner la désignation des pièces 15, 16 et
17.}
\subparagraph{}\textit{Quel est le rôle des mors rapportés 5 et 6 ?}
\subparagraph{}\textit{Quel est le rôle de la douille 13 ?}
\subparagraph{}\textit{Quel est le rôle de la rondelle 12 ?}
\subparagraph{}\textit{Quel est le rôle du boulon 19 dont la tête carrée est à l’intérieur d’une rainure circulaire ?}
\subparagraph{}\textit{Définir les sous-ensembles cinématiques : pour cela colorier avec des couleurs différentes le dessin
d’ensemble.}
\subparagraph{}\textit{Définir les liaisons entre ces sous-ensembles : faites sous la forme [Couleur1/Couleur2 a liaison pivot
par exemple].}
\subparagraph{}\textit{Tracer le schéma cinématique minimal en représentation plane.}
\end{multicols}


\begin{center}
\includegraphics[height=\textheight]{images/plan}
\end{center}

\end{document}


