\documentclass[10pt,fleqn]{article} % Default font size and left-justified equations
\usepackage[%
    pdftitle={CIN : Cinématique du solide},
    pdfauthor={Xavier Pessoles}]{hyperref}
    
\input{style/new_style}
\input{style/macros_SII}

\usepackage{multicol}
\fichetrue
%\fichefalse

%\proftrue
\proffalse

\tdtrue
%\tdfalse

\courstrue
\coursfalse

\def\discipline{Sciences \\Industrielles de \\ l'Ingénieur}
\def\xxtete{Sciences Industrielles de l'Ingénieur}

\def\classe{PTSI}
\def\xxnumpartie{Cycle 3}
\def\xxpartie{Analyse et modélisation des contacts entre solides}

\def\xxnumchapitre{Chapitre 1}
\def\xxchapitre{Présentation des outils de la communication technique}%Étude des chaînes fermées : Détermination des lois Entrée -- Sortie}

\def\xxtitreexo{Actionneur de vanne}
\def\xxsourceexo{\hspace{.2cm} D'après ressources de JP Pupier.}


\def\xxposongletx{2}
\def\xxposonglettext{1.45}
\def\xxposonglety{20}
\def\xxonglet{Part. 3 -- Ch. 1}

\def\xxactivite{Colle 7}
\def\xxauteur{\textsl{Xavier Pessoles}}

\def\xxcompetences{%
\textsl{%
%\textbf{Savoirs et compétences :}\\
%%\noindent \textbf{Résoudre :} à partir des modèles retenus :
%\begin{itemize}[label=\ding{112},font=\color{ocre}] 
%\item \textit{Mod2 -- C12} : modélisation cinématique des liaisons entre solides;
%\end{itemize}
}}

\def\xxfigures{
\includegraphics[width=.8\textwidth]{images/fig_01}
}%figues de la page de garde

\def\xxpied{%
Cycle 3 -- Analyse et modélisation des contacts entre solides \\
Ch. 1 : Présentation des outils de la communication technique -- \xxactivite%
}


\setcounter{secnumdepth}{5}
%---------------------------------------------------------------------------


\begin{document}
%\chapterimage{png/Fond_Cin}
\input{style/new_pagegarde}
\vspace{7cm}
\pagestyle{fancy}
\thispagestyle{plain}


\def\columnseprulecolor{\color{ocre}}
\setlength{\columnseprule}{0.4pt} 

\ifprof
\begin{center}
%\includegraphics[width=.8\linewidth]{images/corr_01}
%\includegraphics[width=.8\linewidth]{images/corr_02}
\end{center}

\else
\begin{multicols}{2}
\subsection*{Mise en situation}
Dans les industries agro-alimentaires, chimiques, et pétrolières il est nécessaire de transvaser des produits liquides, pâteux ou pulvérulents.  Le transfert de ces produits est réalisé par un réseau de conduites sur lesquelles sont placées des vannes à commande manuelle ou motorisée.

Ces vannes ou robinets ont pour fonction de réguler, d'interrompre ou de rétablir les écoulements dans les conduites et cela avec la garantie d'une étanchéité totale et durable.


L’actionneur ACTO 31H de la Société AMRI permet de motoriser les vannes. Il est alimenté par une pression de  60 bars ($1\,\text{bar} = 0,1\,\text{N}/\text{mm}^2 = 10^5\,\text{Pa}$) et il agit sur le carré d'entraînement solidaire du papillon qui fait office d'obturateur de la vanne. 


\begin{center}
\includegraphics[width=\linewidth]{images/fig_02}
\end{center}

Les caractéristiques générales de cet actionneur en font, selon le constructeur, un produit tout particulièrement adapté pour :
\begin{itemize}
\item assurer la rotation du papillon d’un quart de tour de la position fermée à la position ouverte et réciproquement; 
\item interrompre très progressivement le débit afin d’éviter les coups de bélier générateurs de surpressions dangereuses pour les conduites;
\item fournir un couple moteur $C_m$ plus important au voisinage de la  position fermée : en effet, pour ces positions la composante principale du couple résistant $C_r$ augmente lorsque le papillon déforme la bague en  élastomère qui fait office de joint d’étanchéité;
\item assurer un verrouillage mécanique en position fermée.
\end{itemize}



\subsection*{Travail demandé}
\subparagraph{}
\textit{Réaliser à main levée et en 3 dimensions les pièces 6 et 9.}

\subparagraph{}
\textit{Colorier les pièces par classes d'équivalence cinématique.}


\subparagraph{}
\textit{Donner le schéma cinématique de l'actionneur de vanne.}
\end{multicols}

\begin{center}
\begin{tabular}{|c|c|p{.3\linewidth}||c|c|p{.3\linewidth}|}
\hline
18	&3	&Vis CHc M8 &&&\\ \hline
35	&1	&Papillon	&17	&1&	Index\\ \hline
34	&1	&Adaptateur	&16	&1&	Moyeu d'index\\ \hline
33	&1	&Flasque de guidage	&15&	1&	Hublot\\ \hline
32	&1	&Joint torique	&14&	1&	Chapeau\\ \hline
31	&1	&Joint torique	&13&	1&	Joint torique\\ \hline
30	&1	&Cylindre	&12&	4&	Vis H M 14\\ \hline
29	&1	&Joint torique	&11&	2&	Rondelle\\ \hline
28	&1	&Joint torique	&10&	1&	Ecrou Nylstop\\ \hline
27	&1	&Joint torique	&9&	1&	Noix\\ \hline
26	&1	&Coussinet de guidage	&8&	1&	Appui\\ \hline
25	&2	&Joint à lèvres 	&7&	1&	Jonc\\ \hline
24	&1	&Corps de vanne	&6&	1&	Fourche\\ \hline
23	&1	&Bride de platine de vanne	&5&	2&	Galet\\ \hline
22	&12	&Ecrou H	&4&	1&	Mandrin\\ \hline
21	&12	&Goujon	&3&	2&	Bielle\\ \hline
20	&1	&Piston	&2&	1&	Tige de piston\\ \hline
19	&3	&Rondelle Grower	&1&	1&	Carter\\ \hline \hline
Rep.& 	Nomb.	&Désignation	&Rep.	&Nomb.	&Désignation\\ \hline
\end{tabular}
\end{center}


\begin{center}
\includegraphics[width=.8\linewidth]{images/fig_03}
\end{center}
\fi
\end{document}


